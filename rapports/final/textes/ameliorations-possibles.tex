\section{Côté serveur}

	\subsection{Sécurité de l'application} % (fold)
	\label{sub:securite_de_l_application}

		À l'heure actuelle, l'application n'implémente aucune solution de sécurité. Il est donc impossible de garantir l'authenticité d'une requête. Cela pose de graves problèmes en jeu : un joueur pourrait récupérer les conversations d'autres joueurs, ou encore, envoyer des ordres aux armées adverses. De nombreuses solutions de sécurité existent, plus ou moins rapides à mettre en place et plus ou moins fiables.

		\paragraph{Sécurité par l'IP} % (fold)
		\label{par:securite_par_l_ip}

			Une solution très basique et très rapide à mettre en place serait de sauvegarder l'adresse IP du client créant un joueur et de n'autoriser que cette adresse IP à envoyer des ordres aux armées du joueur ou accèder à ses conversations. L'avantage de cette solution est qu'elle ne nécessite aucune modification au niveau du client car l'adresse IP est un identifiant unique géré de façon transparente par le réseau. Le désavantage est le problème du NAT (plusieurs clients avec des adresses IP privées derrière une adresse IP publique), il serait donc impossible de jouer en famille derrière une \enquote{box} par exemple. Cette solution n'est donc pas optimale.

		% paragraph securite_par_l_ip (end)

		\paragraph{Session} % (fold)
		\label{par:session}

			PHP gère nativement les \texttt{SESSION}. À chaque connexion d'un utilisateur, le serveur va lui attribuer un identifiant de session choisi aléatoirement. Pendant tout le déroulement d'une partie, le serveur va pouvoir enregistrer des informations associées à cet identifiant et y accèder facilement. Le client va envoyer dans les en-tête HTTP son identifiant de session à chaque requête afin d'être identifié. L'avantage de cette solution est la possibilité d'avoir plusieurs joueurs avec la même adresse IP ainsi qu'une relative simplicité. Les sessions sont gérées nativement pas les navigateur web mais nécessite quelques modifications au niveau d'autres types de clients. Le principal désavantage des sessions est leur aspect temporaire, en effet, une session (tout comme les cookies en web par exemple) ont une date d'expiration. Même si cette dernière peut-être allongée, une session sera automatiquement terminée à la fermeture du navigateur par exemple.

		% paragraph session (end)

		\paragraph{Jeton d'authentification} % (fold)
		\label{par:jeton_d_authentification}

			Le jeton d'authentification fonctionne de la même manière que les sessions. Il est cependant géré par l'application et non par le langage. Lors de la création d'un utilisateur, un jeton unique est généré et envoyé à l'utilisateur qui l'utilisera par la suite pour effectuer ses requêtes. Cette solution présente les avantages de la session sans la temporalité (car géré côté application serveur et côté application cliente). Le désavantage est une complexité plus élevée lors de la mise en place.

		% paragraph jeton_d_authentification (end)

		\paragraph{Authentification par compte utilisateur} % (fold)
		\label{par:authentification_par_compte_utilisateur}

			La solution de sécurité optimale est l'inscription de l'utilisateur via un nom d'utilisateur et un mot de passe. Cette procédure est plus lourde pour l'expérience utilisateur qui devra s'inscrire avant de pouvoir jouer. La communication des accès et des droits entre les différents acteurs du système (utilisateur, client et serveur) peut être gérée par le protocole OAuth2 (protocole relativement simple à mettre en place car de nombreuses implémentations existent pour tous les langages). Cette solution est la plus lourde à mettre en place à cause de la création des comptes utilisateurs au niveau client et serveur mais apporte la plus grande sécurité. C'est la solution retenu par la grande majorité des API disponibles sur Internet (Dropbox, Facebook, Google\dots).

		% paragraph authentification_par_compte_utilisateur (end)

	% subsection securite_de_l_application (end)

	\subsection{Performances} % (fold)
	\label{sub:performances}

		Les performances actuelles de l'application sont plutôt bonnes. Le langage PHP est un langage léger et rapide. Le framework Laravel ajoute une certaine lourdeur au profit d'une simplicité de développement mais reste très rapide à l'exécution.\\

		Malgré tout, il est possible d'améliorer certaines requêtes SQL. En effet, dans un objectif de rapidité du développement, un ORM a été utilisé. Les performances de l'ORM Eloquent sont très bonnes concernant les requêtes classiques mais il est possible d'optimiser les requêtes pour des traitements plus lourds. Par exemple, la création d'une carte est très longue (plus de 30 secondes de traitement sur un ordinateur portable classique, les temps sont largement inférieurs avec un SSD) car il est nécessaire de créer toutes les cases afin de générer des identifiants uniques avec l'incrémentation automatique de la base de données puis d'ajouter dans une table pivot les relations de voisinage entre toutes les cases. Plusieurs améliorations sont ici possibles.

		\paragraph{Traitement de la requête en SQL} % (fold)
		\label{par:traitement_de_la_requete_en_sql}

			Nos connaissances en SQL sont limitées mais il devrait être possible d'optimiser ces nombreuses insertions en les exécutant d'une seule fois en SQL.

		% paragraph traitement_de_la_requete_en_sql (end)

		\paragraph{Amélioration de la conception} % (fold)
		\label{par:amelioration_de_la_conception}

			L'amélioration possible au niveau de la conception concerne les liaisons doubles entre les cases. En effet, il n'est pas forcément nécessaire de sauvegarder les deux liens entre les cases car les liaisons sont obligatoirement bidirectionnelles. Nous pourrions uniquement sauvegarder les liaisons par ordre croissant (lien de 1 vers 2 mais pas de 2 vers 1).

		% paragraph amelioration_de_la_conception (end)

		\paragraph{Type de base de données} % (fold)
		\label{par:type_de_base_de_donnees}

			Il est possible d'améliorer les performances de la base de données en traitant les cases dans une base de données NoSQL orientée graphe par exemple. Le schéma serait dans ce cas plus adapté. Si le choix d'une base de données NoSQL peut sembler extrémiste, il est aussi possible de se renseigner au niveau des types spécifiques aux données géographiques dans les bases de données relationnelles (comme par exemple PostGIS).

		% paragraph type_de_base_de_donnees (end)

	% subsection performances (end)

	\subsection{Séparation des services} % (fold)
	\label{sub:separation_des_services}

		Actuellement le serveur est seul pour assurer 3 services : l'application métier, la base de données et le serveur de queues. Afin d'améliorer les performances et la maintenabilité de l'application, il serait préférable de séparer ces trois services sur trois machines physiques différentes ou au minimum sur trois machines virtuelles. L'avantage de cette séparation est la possibilité de maintenir différents composants plus simplement (possibilité de mettre à jour certains logiciels d'une machine sans affecter les autres services). L'avantage supplémentaire des trois machines physique est de répartir la charge de l'application.

	% subsection separation_des_services (end)

	\subsection{Fonctionnalités} % (fold)
	\label{sub:fonctionnalites}

		\paragraph{Ajout des cases maritimes} % (fold)
		\label{par:ajout_des_cases_maritimes}
			L'application ne gère actuellement pas les cases maritimes. La conception permet de gérer ce type de case de manière simple. Les plus grosses modifications concerneront les ordres avec les nouvelles règles introduites par la présence de cases maritimes.
		% paragraph ajout_des_cases_maritimes (end)

		\paragraph{Proposition de partie} % (fold)
		\label{par:proposition_de_partie}
			Il serait intéressant de pouvoir ajouter un endpoint à l'API afin de proposer des parties. Cette fonctionnalité pourrait permettre de récupérer les parties les plus intéressantes (place disponible, proche d'être remplie\dots) pour le joueur afin de pouvoir ensuite les rejoindre.
		% paragraph proposition_de_partie (end)

	% subsection fonctionnalites (end)

\section{Côté client}

	 \subsection{Interface utilisateur} % (fold)
	 \label{sub:interface_utilisateur}
	 
	 % subsection interface_utilisateur (end)

	 \subsection{Génération automatique des modèles} % (fold)
	 \label{sub:g_n_ration_automatique_des_mod_les}
	 
	 % subsection g_n_ration_automatique_des_mod_les (end)

	 \subsection{Refactor de la communication serveur} % (fold)
	 \label{sub:refactor_de_la_communication_serveur}
	 
	 % subsection refactor_de_la_communication_serveur (end)