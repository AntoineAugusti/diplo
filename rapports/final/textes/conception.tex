\section{Diagramme de cas d'utilisation}
	\begin{figure}[!h]
		\centering
		\includegraphics[scale=0.4]{diagrammes/use_case.png}
		\caption{Diagramme de cas d'utilisation}
	\end{figure}

\newpage

\section{Diagramme de classes participantes}
	\subsection{Partie Serveur}
		\subsubsection{Moteur du jeu}
        Les diagrammes décrivant le moteur du jeu sont les figures \ref{pparties}, \ref{parmees}, \ref{pcartes}, \ref{pjoueurs}, \ref{pmessages}.
			\begin{figure}[!h]
				\centering
                \includegraphics[scale=0.4]{diagrammes/php-export-cut/Parties.png}
                \caption{\label{pparties}Package Parties}
			\end{figure}

			\begin{figure}[!h]
				\centering
                \includegraphics[scale=0.4]{diagrammes/php-export-cut/Armees.png}
                \caption{\label{parmees}Package Armees}
			\end{figure}
            
			\begin{figure}[!h]
				\centering
                \includegraphics[scale=0.25]{diagrammes/php-export-cut/Cartes.png}
                \caption{\label{pcartes}Package Cartes}
			\end{figure}

			\begin{figure}[!h]
				\centering
                \includegraphics[scale=0.4]{diagrammes/php-export-cut/Joueurs.png}
                \caption{\label{pjoueurs}Package Joueurs}
			\end{figure}

			\begin{figure}[h!]
				\centering
                \includegraphics[scale=0.32]{diagrammes/php-export-cut/Messages.png}
                \caption{\label{pmessages}Package Messages}
                \pagebreak
			\end{figure} 

        \newpage
        \subsubsection{Les phases du jeu}
        Les diagrammes décrivant les phases du jeu sont les figures \ref{pphases}, \ref{pevents}.

			\begin{figure}[!h]
				\centering
                \includegraphics[scale=0.5]{diagrammes/php-export-cut/Phases.png}
                \caption{\label{pphases}Package Phases}
			\end{figure}

			\begin{figure}[!h]
				\centering
                \includegraphics[scale=0.4]{diagrammes/php-export-cut/Events.png}
                \caption{\label{pevents}Package Events}
			\end{figure}
        
        \newpage
        \subsubsection{Les controleurs}
        Les diagrammes décrivant les phases du jeu sont les figures \ref{phandlers}, \ref{pcommands}, \ref{pconsole}, \ref{phmiddleware}.
        Les diagrammes décrivant le package \verb|Http.Controllers| étant trop
        grand, il est consultable à l'url \url{http://diplo-le-jeu.fr/images/Http\_Controllers.png}

			\begin{figure}[!h]
				\centering
                \includegraphics[scale=0.25]{diagrammes/php-export-cut/Handlers.png}
                \caption{\label{phandlers}Package Handlers}
			\end{figure}

			\begin{figure}[!h]
				\centering
                \includegraphics[scale=0.5]{diagrammes/php-export-cut/Commands.png}
                \caption{\label{pcommands}Package Commands}
			\end{figure}

			\begin{figure}[!h]
				\centering
                \includegraphics[scale=0.5]{diagrammes/php-export-cut/Console.png}
                \caption{\label{pconsole}Package Console}
			\end{figure}

			\begin{figure}[!h]
				\centering
                \includegraphics[scale=0.5]{diagrammes/php-export-cut/Http_Middleware.png}
                \caption{\label{phmiddleware}Package Http.Middleware}
			\end{figure}

        \newpage
        \subsubsection{Le framework}
        Les diagrammes décrivant l'utilisation du framework laraval sont les figures \ref{pproviders}
			\begin{figure}[!h]
				\centering
                \includegraphics[scale=0.25]{diagrammes/php-export-cut/Providers.png}
                \caption{\label{pproviders}Package Providers}
			\end{figure}
        \newpage

        \newpage
		\subsubsection{Les ordres}
        Les diagrammes décrivant les ordres du jeu étant trop grand, ils sont
        consultables à l'url suivante \url{http://diplo-le-jeu.fr/images/Ordres.png}

        \newpage
		\subsubsection{Les exceptions}
        Les diagrammes décrivant les exceptions du jeu étant trop grand, ils
        sont consultables à l'url suivante \url{http://diplo-le-jeu.fr/imaes/Exceptions.png}

        
    \pagebreak
	\subsection{Partie Client}

    \newpage
\section{Diagramme de packages}
	\subsection{Partie Serveur}
        Le diagramme de packages étant trop grand, ils sont consultables à l'url
        suivante \url{http://diplo-le-jeu.fr/images/Diagramme-Packages.png}

		\newpage

		\subsubsection{Deuxième partie}
			\vspace{10mm}
			\begin{figure}[!h]
				\centering
				\includegraphics[scale=0.4]{images/DP2.png}
				\caption{Le package Ordres}
			\end{figure}

		\subsubsection{Troisième partie}
			\vspace{10mm}
			\begin{figure}[!h]
				\centering
				\includegraphics[scale=0.4]{images/DP3.png}
				\caption{Le package Exceptions}
			\end{figure}
		\newpage
	\subsection{Partie Client}
\section{Diagramme de séquence système}
	\subsection{Créer une partie}
		\vspace{10mm}
		\begin{figure}[!h]
			\centering
			\includegraphics[scale=0.5]{diagrammes/dss_create.png}
			\caption{Créer une partie}
		\end{figure}


	\subsection{Jouer son tour}
		\vspace{10mm}
		\begin{figure}[!h]
			\centering
			\includegraphics[scale=0.3]{diagrammes/gameplay.png}
			\caption{Jouer son tour}
		\end{figure}
		\vspace{70mm}

\section{Sources du DCP et du diagramme de package}
	\begin{itemize}
		\item DCP : \url{http://goo.gl/7uLQP4}
		\item Diagramme de Package : \url{http://goo.gl/LsiHrs}
	\end{itemize}
