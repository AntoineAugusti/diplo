Afin de travailler plus efficacement, nous avons choisi de diviser l'équipe en 2. Ainsi Antoine Augusti, Étienne Batise et Thibaud Dauce ont travaillé sur le logiciel serveur et Manon Ansart, Jean-Claude Bernard et Nathan Malo ont travaillé sur le logiciel client.\newline

\section{Côté serveur}
    Pour le logiciel serveur, nous avons déterminé plusieurs grandes parties :
	\begin{enumerate}
		\item La spécification des endpoints de l'API;
		\item La mise à jour de la conception pour s'intégrer au framework Laravel;
		\item Le développement.\newline
	\end{enumerate}

    Par ailleurs, le framework Laravel permet un grande indépendance de certaines parties du programme. Nous avons donc également divisé le développement selon les parties indépendantes.
	\begin{enumerate}
		\item \textbf{La couche de stockage.} Il s'agit d'utiliser le service de migration et l'ORM fournis par Laravel afin de faciliter la synchronisation avec la base de données.
		\item \textbf{La logique métier.} Elle réalise le cœur de métier de l'application, elle n'est pas absolument pas liée au protocole HTTP.
		\item \textbf{Les contrôleurs.} À partir des requêtes HTTP reçues, en déléguant le travail à la logique métier, ils renvoient des réponses conformes aux spécifications de l'API.
	\end{enumerate}\bigskip

    Nous avons donc réparti les taches de la manière suivante :
	\begin{description}
        \item [Antoine Augusti] : Spécification de l'API, mise à jour de la conception, développement de la logique métier et des contrôleurs;
		\item [Etienne Batise] : Spécification de l'API, travail sur la couche de stockage;
		\item [Thibaud Dauce] : Spécification de l'API, mise à jour de la conception, développement de la logique métier et des contrôleurs.
	\end{description}


\section{Côté client}
	Pour le logiciel client, nous avons déterminé trois grandes parties :
	\begin{enumerate}
		\item l'interface utilisateur
		\item le moteur du jeu et ses exceptions
		\item la communication serveur et ses exceptions
	\end{enumerate}

	Ainsi, Manon Ansart a réalisé la partie moteur de jeu du client, contenu dans le package \texttt{MoteurJeu}. Jean-Claude Bernard a développé la partie interface utilisateur du client, du package \texttt{Affichage}, et Nathan Malo a fait la partie communication serveur du package \texttt{Reseau}. Le fichier java principal \enquote{main} est réalisé par Jean-Claude Bernard.
