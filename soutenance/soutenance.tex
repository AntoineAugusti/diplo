% L'option handout permet de supprimer la barre de navigation
\documentclass[handout]{beamer}
\usepackage[utf8]{inputenc}
\usepackage[french]{babel}
\usepackage[T1]{fontenc}
\usepackage{amsmath}
% Pour pouvoir insérer des images
\usepackage{graphicx}
\usepackage{wrapfig}
\graphicspath{images/}
% Gestion des couleurs
\usepackage{color}
\definecolor{red}{RGB}{231, 76, 60}

% Un joli thème flat
\usetheme{Rochester}

% Personnalisation du thème
\usecolortheme[named=red]{structure}
% Numéro de slides dans le footer
\setbeamertemplate{footline}[frame number]
\setbeamertemplate{blocks}[shadow=false]

% ------------------------------------ %
% -- METADONNÉES DU DOCUMENT --------- %
\title{
	Projet Informatique Répartie - Diplo
}
\author{
	\textsc{Ansart}, \textsc{Augusti}, \textsc{Bernard},
	\textsc{Batise}, \textsc{Dauce}, \textsc{Malo}
}
\date{12 mai 2015}

% Début du document
\begin{document}

	% Génération de la page de titre
	\begin{frame}[plain]
		\titlepage
	\end{frame}

	% Génération du sommaire
	\begin{frame}[plain]
		\frametitle{Sommaire}
		\tableofcontents
	\end{frame}

	\section{Choix techniques}
		\subsection{Architecture générale}
	\begin{frame}
		\frametitle{Architecture générale}
		\begin{figure}
			\centering
			\includegraphics[width=1\textwidth]{images/architecture.png}
			\caption{Architecture générale}
			\label{fig:architecture-generale}
		\end{figure}
	\end{frame}

\subsection{Serveur}
	\begin{frame}
		\frametitle{API REST}
		\begin{itemize}
			\item Une API respectant les bonnes pratiques REST et HTTP;
			\item Entièrement en JSON;
			\item Une documentation claire et complète (\url{https://developers.diplo-lejeu.fr});
			\item Organisation en trois ressources : Conversations, Ordres, Parties;
			\item Documentation écrite en Markdown puis affichée sur une interface web.
		\end{itemize}
	\end{frame}

	\begin{frame}
		\frametitle{Composants}
		\begin{itemize}
			\item Framework MVC PHP : Laravel 5;
			\item ORM : Eloquent, intégré à Laravel;
			\item Base de données relationnelle : SQLite;
			\item Push / Pull Queue : IronMQ via \texttt{iron.io};
			\item Agrégateur d'exceptions : Bugsnag.
		\end{itemize}\bigskip
		Le tout hébergé sur un VPS chez RunAbove, à Roubaix. DNS et certificats SSL gérés par CloudFlare
	\end{frame}

\subsection{Client}
	\begin{frame}
		\frametitle{Client}
		\begin{itemize}
			\item Consommation d'API REST : utilisation de la classe \texttt{HTTPSUrlConnection}; \newline
			\item Gestion de l'affichage : CLI avec introspection; \newline
			\item Moteur du Jeu : mise en relation des différents éléments du jeu.
		\end{itemize}
	\end{frame}


	\section{Problèmes rencontrés}
		\subsection{Serveur}
	\begin{frame}
		\frametitle{Serveur}
        \begin{itemize}
            \item Reconception globale du programme \newline
            \item Rédaction du protocole de communication \newline
            \item Manque de tests complets \newline
        \end{itemize}
	\end{frame}

\subsection{Client}
	\begin{frame}
		\frametitle{Client}
	\end{frame}


	\section{Améliorations possibles}
		\subsection{Serveur}
	\begin{frame}
		\frametitle{Sécurité de l'application}
		\begin{itemize}
			\item Sécurité par l'IP ;
			\item Session ;
			\item Jeton d'authentification ;
			\item Authentification par compte utilisateur avec OAuth2.
		\end{itemize}
	\end{frame}

	\begin{frame}
		\frametitle{Protocole OAuth2}
		\begin{figure}[H]
			\centering
			\includegraphics[scale=0.3]{images/oauth2.png}
			\caption{http://www.ibuildings.nl/blog/2013/03/secure-your-rest-api-oauth2-implicit-grant}
		\end{figure}

	\end{frame}

	\begin{frame}
		\frametitle{Autres améliorations}
		\begin{itemize}
			\item Performances :
				\begin{itemize}
					\item Amélioration des requêtes SQL ;
					\item Déléguer les requêtes à la base de données (fonction SQL) ;
					\item Changer de type de base de données (Neo4j ou PostgreSQL).
				\end{itemize}
			\item Séparation des services :
				\begin{itemize}
					\item Machines physiques ;
					\item Machines virtuelles.
				\end{itemize}
			\item Fonctionnalités :
				\begin{itemize}
					\item Cases maritimes ;
					\item Proposition de parties.
				\end{itemize}
		\end{itemize}
	\end{frame}

\subsection{Client}
	\begin{frame}
		\frametitle{Client}
		\begin{itemize}
			\item Amélioration de l'interface graphique ;
			\item Carte plus générique ;
			\item Refactor de la communication avec l'API ;
			\item Sérialisation / désérialisation des modèles grâce au JSON.
		\end{itemize}
	\end{frame}



\end{document}
